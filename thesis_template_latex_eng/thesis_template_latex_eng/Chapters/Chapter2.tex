% Chapter 2

\chapter{Related Work} % Write in your own Chapter title
\label{Chapter2}

\section{Richnote} 
As described in the introduction, richnote\cite{Richnote} is a algorithm suggested to best solve the problem of delivering content rich multimedia notifications scaled down based on constraints on the device they are to be displayed on.

I'd like to expand on the problem.

An example of the problem would be if you are using Spotify, and your close friend starts listening to a new song. You could get a notification about this song including a full 30 second preview of the song. Now your device has to download this notification, this uses battery on your device. If your device has full battery, downloading and playing is not a problem. However, if the device is very low on battery, you might loose precious last battery power and end with a dead device a long way from a charger. You might not be able to show your ticket on the bus home(where you can charge). So to address this, one would have to manage the notification. 

Another part of the problem is available bandwidth. You don't want to use the final bandwidth of the month on a 30 second preview of a song on Spotify, when you rely on it to purchase a digital ticket for the bus home. Then you would have to buy additional data or end up with a long walk.

attention, use of time, irritating amount of notifications

solution suggested in richnote document

breakdown of algorithm, problem to execute on device.

server (site other students thesis) -> client implementation(this thesis)